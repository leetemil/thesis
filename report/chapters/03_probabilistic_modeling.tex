\chapter{Probabilistic Modeling}
\todo{find better chapter title?}
\todo{brief chapter intro}


\section{Inference}
In statistics, \textit{inference} is the process of finding out information or even determining the underlying distribution of data. In our settings, we care about inferring the distribution that governs the observed data. This is intractable to compute directly in many cases of interest, so usually the best alternative is to approximate.

There is at least one kind of inference type that we desire in this project context: the marginal inference, i.e. the marginal distribution of a variable. In our case, we care about the marginal distribution 
\begin{equation}
    p(\ve{x}) = \int_{\ve{z}} p(\ve{x} \mid \ve{z}) p(\ve{z}) \diff \ve{z} \label{eq:marginal_distribution}
\end{equation}
for samples $\ve{x}$ and possible representations $\ve{z}$. This expression of $p(\ve{x})$ is also called a mixture model, as we integrate over all possible values of $\ve{z}$, adding a weighted conditional distribution to $p(\ve{x})$ for each value. Defining $p(\ve{x})$ in these terms also links the type of inference to \textit{Bayesian inference}, that incorporates proposed knowledge about the system as a prior distribution using Bayes' theorem
\begin{equation}
    p(\ve{z} \mid \ve{x}) = \frac{p(\ve{x} \mid \ve{z}) p(\ve{z})}{p(\ve{x})}. \label{eq:bayes_theorem}
\end{equation}
This relation will become more apparent in section \ref{sec:elbo}.
% Secondly, we wish to infer the maximum a posteriori probability 

\subsection{Variational Inference}
\label{sec:variational_inference}
In \textit{variational inference}, an intractable distribution $p$ is approximated by inspecting a set of candidate distributions $\mathcal{Q}$ and choosing a distribution $q \in \mathcal{Q}$ which is the most similar to $p$. This way, we can use $q$ as a substitute approximation for $p$, using it for whatever task that required $p$.

Similarity in our case is determined by the \textit{Kullback-Leibler divergence} (KL), which is roughly a measure of distance between two distributions: KL divergence is zero when the distributions are identical and positive otherwise, but not necessarily symmetric (which one would expect from a distance measure) \cite{standfordVAE}. It is defined as follows:
\[ \kl{q}{p} =  \expectedvalue_{\ve{x}\sim q}\brts{\log q - \log p}.\]
Note that the expectation is with respect to the first input of the divergence. Variational inference is used in variational autoencoders (VAEs) to simultaneously (1) derive an approximation $q\prts{\ve{z} \mid \ve{X}}$ to the true conditional distribution $p\prts{\ve{z} \mid \ve{x}}$, where $\ve{x}$ is a sample from the observed data space, and $\ve{z}$ is a representation given that sample and (2) increase the loglikelihood of the observed data. The first is achieved by minimizing the KL divergence. Here the procedure is explained, but not how the divergence measure is used to find the most similar candiate distribution for $p$. The actual variational inference process maximizes the \textit{evidence lower bound} (ELBO). This key component of the VAE is discussed further in section \ref{sec:elbo}.

\section{Autoencoders}
\label{seq:autoencoders}
An \textit{autoencoder} network is reconstructing its input from an internal representation of the input. To make the network non-trivial (and useful), it is constrained in its capacity to fully represent inputs. This has the effect that it forces the network to compress inputs into its most important properties in order for the network to make the most accurate reconstruction. As a result of this important side effect, meaningful representations of the inputs can be learned by the autoencoder, making it useful in unsupervised settings where only the raw inputs are available.

Such a model is typically constructed by two networks: the \textit{encoder} and the \textit{decoder} network. The task of the encoder is to convert the input into its compressed internal representation (as the name suggests, encoding the input). The output of the encoder is then passed as input to the decoder, tasked to reconstruct the original input of the encoder.

This idea can be expanded to probabilistic settings: \todo{expand. ref bayesian neural networks section}

An important consequence is that the autoencoder will learn to represent samples from the input data distributions well, while samples that are unlikely will not. This is important as is relaxes the needed power of the network, only requiring it to accurately represent samples that are likely under the data distribution.
\section{The Variational Autoencoder}

One way to utilize the autoencoding setup is to combine it with variational inference: Representations are latent variables $\ve{z}$ from a latent space of dimension $d$. Given a prior distribution $p\prts{\ve{z}}$ and a dataset $\mathcal{D}$ containing samples $\ve{x}$ from some sample space, we seek to build an \textit{encoder} neural network that can transform $\ve{x}$ into a latent variable $\ve{z}$. Another \textit{decoder} neural network then learns to transform latent variables back to sample-space by inference. This entire setup is collectively called a \textit{variational auto-encoder} (VAE).
VAE's have at least two major useful properties: (1) they learn latent representations of its inputs and (2) they contain a generative model. The generative model is a joint distribution on $\ve{x}$ and $\ve{z}$ that can be expressed as $p(\ve{x}, \ve{z}) = p(\ve{x} \mid \ve{z}) p(\ve{z})$. The decoder correspond to the conditional distribution $p(\ve{x} \mid \ve{z})$. The encoder is the part of the model that approximates $p(\ve{z} \mid \ve{x})$ by a distribution $q(\ve{z} \mid \ve{x})$. 

In order to make the generative model likely to generate samples that are similar to the ones observed in $\mathcal{D}$, the task is to maximize the probability $\px$ of the observed samples $\ve{x} \in \mathcal{D}$. That is, if the model has high probability on the observed samples, then samples that are similar should be probable as well; conversely, samples that are unlike what has been observed are unlikely to be generated by the model. This is achieved by integrating out the latent variables $\ve{z}$ from the generative model $p(\ve{x}, \ve{z})$ to get the marginal distribution $\px$ as expressed in equation (\ref{eq:marginal_distribution}).

This requires two things, namely making a suitable latent space, and integration over that space to get $\px$. The former is learned by the network (how this is done is discussed in section \ref{sec:elbo} and \ref{seq:gradient_based)optimization}). The latter is usually intractable, so the solution is to instead approximate the integral by sampling a number of latent variables $(\ve{z}_1, \ldots, \ve{z}_n)$ from the latent space and use $\frac{1}{n}\sum_{i = 1}^n p(\ve{x} \mid \ve{z}_i)$ as an approximation for $\px$. The problem with this approach is that if we sample from $\pz$ we might need many samples to accurately approximate $\px$ because for almost all $\ve{z}$, the probability $\pxz$ is very small. We preferably want to sample only values of $\ve{z}$ for which $\pxz$ is large, as these are the values significant to the approximation. Such a distribution over $\ve{z}$ would allow us to skip all the insignificant samples, effectively reducing the amount of samples needed to approximate $\px$. In fact, the distribution we want to sample $\ve{z}$ from is $\pzx$, determining the most likely values of $\ve{z}$ given a specific sample $\ve{x}$. This leads us to $\q$, the encoding part of the VAE, as this is the model we use to find candidate distributions similar to $\pzx$.

In order to determine a suitable approximation distribution $\q$ and to train the autoencoder network, an objection function is needed with respect to the network parameters. In alignment with existing literature, we denote $\ve{\theta}$ as the parameters of the generative model (decoder \todo{is this correct?}), and $\ve{\phi}$ as the parameters of the variational model (encoder). As discussed in section \ref{sec:variational_inference}, the ELBO is what the network will maximize.

\subsection{Evidence Lower Bound}
\label{sec:elbo}
The evidence lower bound (ELBO) is the measure used to train the variational autoencoder. Without further explanation the ELBO is
\begin{equation}
    \mathcal{L}(\ve{x}) = \expectedvalue_{\ve{z}\sim q}\brts{\log \pxz} - \kl{\q}{\pz}\label{eq:elbo},
\end{equation}
where the expectation subnotation $\ve{z} \sim q$ is shorthand for $\ve{z} \sim \q$ (used throughout this section). It contains two expressions, an expectation of $\log \pxz$ and a divergence between $\q$ and $\pz$. Note that if the prior distribution on $\ve{z}$ is assumed unit Gaussian, $\pz = \mathcal{N}(\ve{0}, \ve{I})$ and if $\q$ is also Gaussian, then the divergence term can be computed analytically.

The expression in (\ref{eq:marginal_distribution}) can be derived in several ways; one is to regard the goal of the variational inference, namely to find a suitable distribution $\q$ that approximates $\pzx$. To determined suitability the KL divergence is used. From this expression, we get the following derivation:
\begin{align}
    \kl{\q}{\pzx} &= \expectedvalue_{\ve{z}\sim q}\brts{\log \q - \log \pzx}\\
    &= \expectedvalue_{\ve{z}\sim q}\brts{\log \q - \log \prts{\frac{\pxz \pz}{\px}}}\label{eq:32}\\
    &= \expectedvalue_{\ve{z}\sim q}\brts{\log \q - \log \pxz - \log \pz + \log \px}\\
    &= \expectedvalue_{\ve{z}\sim q}\brts{\log \q - \log \pz}  - \expectedvalue_{\ve{z}\sim q}\brts{\log \pxz} + \log \px\\
    &= \kl{\q}{\pz}  - \expectedvalue_{\ve{z}\sim q}\brts{\log \pxz} + \log \px\label{eq:35},
\end{align}
from which it follows that
\begin{align}
    \log \px &\ge \log \px - \kl{\q}{\pzx}\label{eq:36}\\
             &= \expectedvalue_{\ve{z}\sim q}\brts{\log \pxz} - \kl{\q}{\pz}\\
             &= \mathcal{L}(\ve{x})\label{eq:38}.
\end{align}
The derivation uses Bayes' Theorem in equation (\ref{eq:32}) and the definition of KL in (\ref{eq:35}). The inequality in (\ref{eq:36}) is due to the fact that KL is always nonnegative. From equations \ref{eq:36} - \ref{eq:38} we can see why the expression is named a lower bound; it bounds the logarithm of the data (evidence) distribution from below. Because the logarithm is monotonically increasing function, maximizing $\log \px$ will also maximize $\px$. Since $\mathcal{L}(\ve{x})$ is a lower bound, a larger ELBO ensures a better guarantee on how small $\px$ can be. It is also worth noting that the gap between $\px$ and $\mathcal{L}(\ve{x})$ is $\kl{\q}{\pzx}$, indicating that increasing the ELBO effectively decreases the divergence between $\q$ and $\pzx$, which is what we desire in the first place.

In some places the ELBO is denoted as
\[ \mathcal{L}(\ve{x}) = \expectedvalue_{\ve{z}\sim q}\brts{\log \frac{p(\ve{x}, \ve{z})}{\q}},\]
which is equivalent to (\ref{eq:elbo}), as
\begin{align*}
    \expectedvalue_{\ve{z}\sim q}\brts{\log \frac{p(\ve{x}, \ve{z})}{\q}} &= \expectedvalue_{\ve{z}\sim q}\brts{\log p(\ve{x}, \ve{z}) - \log \q}\\
    &= \expectedvalue_{\ve{z}\sim q}\brts{\log \pzx + \log \pz - \log \q}\\
    &= \expectedvalue_{\ve{z}\sim q}\brts{\log \pzx} + \expectedvalue_{\ve{z}\sim q}\brts{\log \frac{\pz}{\q}}\\
    &= \expectedvalue_{\ve{z}\sim q}\brts{\log \pzx} - \expectedvalue_{\ve{z}\sim q}\brts{\log \frac{\q}{\pz}}\\
    &= \expectedvalue_{\ve{z}\sim q}\brts{\log \pzx} - \kl{\q}{\pz}\\
    &= \mathcal{L}(\ve{x}),
\end{align*}
where it is used that $\log a = - \log \frac{1}{a}$ in the fourth line.

\subsection{Gradient Based Optimization}
\label{seq:gradient_based)optimization}
By far the most common way to optimize artificial neural networks is to apply a gradient-based algorithm to iteratively optimze the parameters of the network. This implies that the objective must be differentiable with respect to the optimized parameters and that the gradient can be computed. In the VAE setting, the optimized parameters are those of the encoder, $\ve{\phi}$ and the decoder $\ve{\theta}$. That is, for each input $\ve{x}$ of the network, we wish to compute the gradient $\nabla_{\ve{\phi}, \ve{\theta}} \mathcal{L}(\ve{x})$ of the ELBO with respect to the model parameters $\ve{\phi}$ and $\ve{\theta}$. The objective function of a batch of samples is the sum/average over all the individual ELBOs of the batch samples.

It turns out that computing the gradient of the ELBO is in general intractable \cite{kingma2017variational}, potentially posing an issue with gradient based optimization. There are however unbiased approximations that can substitute the gradient, which allows us to train the model using those approximations and still achieve meaningful results. The gradient $\nabla_{\ve{\theta}} \mathcal{L}(\ve{x})$ with respect to $\ve{\theta}$ can be approximated by $\nabla_{\ve{\theta}}\prts{\log p(\ve{x}, \ve{z})}$ by sampling a value $\ve{z}$ from $\q$ and use this as an approximation of the expected value. This is basically because the expectation is dependent on $q$, which is parameterized by $\ve{\phi}$, so the gradient w.r.t $\ve{\theta}$ can safely be moved inside the expectation \todo{see p. 18 kingma phd equations 2.14 - 2.17}. This is however not something that can be done when taking the gradient w.r.t. $\ve{\phi}$. That is, we have
\[ \nabla_{\ve{\phi}}\mathcal{L}(\ve{x}) = \nabla_{\ve{\phi}} \expectedvalue_{\ve{z}\sim q}\brts{\log \frac{p(\ve{x}, \ve{z})}{\q}} \neq \expectedvalue_{\ve{z}\sim q}\brts{\nabla_{\ve{\phi}} \log \frac{p(\ve{x}, \ve{z})}{\q}}. \]
This means that we cannot directly sample a $\ve{z}$ value and compute a gradient approximation as we did in the gradient approximation w.r.t. $\ve{\theta}$. Nor can we backpropagate through the sampling of $\ve{z}$ because this is a random variable. The way to overcome this is to use a \textit{reparameterization trick} to make the expectation and $\ve{z}$ depend on a noise parameter $\ve{\epsilon}$ instead of $\q$, which depends on $\ve{\phi}$. This parameter can be moved outside the backward gradient pass, allowing for backpropagation through the model.

\subsubsection{Reparameterization Trick}
Instead of sampling $\ve{z}$ from $\q$, we sample an independent random variable $\ve{\epsilon}$ from which we compute $\ve{z}$ as a (deterministic) function $f$ of $\ve{x}$, $\ve{\phi}$ and $\ve{\epsilon}$. This way, the randomness has been moved 'outside' of $\ve{z}$ to an input parameter $\ve{\epsilon}$. Computing $\ve{z}$ is now an evaluation of a deterministic function given its inputs. If $f$ can also be differentiated, we can do compute gradients through $\ve{z}$ w.r.t $\ve{\phi}$ and $\ve{\theta}$. With this reparameterization we have that $\ve{z} = f(\ve{x}, \ve{\phi}, \ve{\epsilon})$. This allows us to write the expectation in $\mathcal{L}(\ve{x})$ over $\ve{\epsilon}$ instead of $\ve{z}' \sim \q$:
\begin{align*}
\nabla_{\ve{\phi}}\mathcal{L}(\ve{x}) &= \nabla_{\ve{\phi}} \expectedvalue_{\ve{z}'\sim q}\brts{\log \frac{p(\ve{x}, \ve{z'})}{q(\ve{z}' \mid \ve{x})}}\\
&= \nabla_{\ve{\phi}} \expectedvalue_{\ve{\epsilon}}\brts{\log \frac{p(\ve{x}, \ve{z})}{\q}} &&\text{by reparameterization}\\
&= \expectedvalue_{\ve{\epsilon}}\brts{\nabla_{\ve{\phi}} \log \frac{p(\ve{x}, \ve{z})}{\q}} &&\text{as the expectation does not depend on $\ve{\phi}$}\\
&\simeq \nabla_{\ve{\phi}} \log \frac{p(\ve{x}, \ve{z})}{\q} &&\text{by sampling $\ve{\epsilon}$.}
\end{align*}
The important takeaway is that this reparameterization trick makes it tractable us to approximate the gradient of the ELBO w.r.t. the model parameters and go gradient based optimization. In our settings, the function $f$ is simply $f(\ve{\mu}, \ve{\sigma}, \ve{\epsilon}) = \ve{\mu} + \ve{\epsilon} \odot \ve{\sigma}$, where $\ve{\mu}$ and $\ve{\sigma}$ are distribution parameters of $\q$ that are determined by $\ve{\phi}$. The operator $\odot$ is the element-wise multiplication. We sample $\ve{\epsilon}$ from a unit Gaussian $\mathcal{N}\prts{\ve{0}, \ve{I}}$, effectively making $\ve{z} \sim \mathcal{N}\prts{\ve{\mu}, \ve{\sigma I}} = \q$, as we desire. This sampling of $\ve{\epsilon}$ implies that this trick is only viable for a continuous latent space (which is our case), as the sampling is continuous. It also reveals the distribution class of $\q$ in our settings to be Gaussian. This is an important aspect, as we can see from equation (\ref{eq:marginal_distribution}) that this decision makes $\px$ a mixture model over Gaussians, if we use $\q$ in place of $\pzx$.

\section{Bayesian Neural Networks}
In the above gradient-based optimization learning procedure, the optimization is applied directly to the encoder and decoder parameters. This application yield \textit{maximum a posteriori} (MAP) estimates of the optimal parameters, if regularization is applied to the \textit{maximum likelihood estimate} in form of a Gaussian prior on the model parameters \cite{blundell2015weight}. These are point estimates to the parameters that maximize the ELBO, but the modeling procedure does not capture the uncertainty inherent in the estimation process. Ideally it is desirable to capture this uncertainty during learning as this information tells us how much we should rely the estimates.

In \textit{Bayesian Neural Networks} (BNN), the parameters $\ve{\theta}$ of the model are not fixed, but distributed according to some distribution $p(\ve{\theta})$, which can be conditioned on the observed data to yield a posterior distribution $p(\ve{\theta} \mid \ve{x})$ that captures both the best estimate (the mean of the posterior) and the uncertainty involved in the estimate (the deviation of the distribution from the mean). This allows us not only to ascertain how precise the estimates are, but also what likely alternatives would be, as we can sample parameters from the distribution. We already applied this idea to the latent variable $\ve{z}$ and there is no significant barrier that prevents us from extending this idea to the parameters of the model itself.

The networks are Bayesian in the sense that the prior and the posterior of the parameters interact with each other as expressed by Bayes' Theorem (\ref{eq:bayes_theorem}). In our case of neural network parameters this computation is intractable, and so variational inference is applied once again to approximate the posterior. So the augmentation of the neural network into a BNN is simply a use of the same probabilistic modeling and inference to a broader set of parameters. Effectively, this means that the model learns the distribution over the parameters and not the parameters themselves, and it learns these distributions by approximation from samples. This way, an ensemble of point-estimate parameter networks is available through sampling from the BNN \cite{blundell2015weight}.

\todo{make a graph figure showing the change in settings}

In our modeling, we applied this extension to the parameters $\ve{\theta}$ of the decoder of our VAE model. See section \todo{reference our practical bayesian VAE section} for the details, namely what the priors on the parameters are, and how this affects the ELBO learning objective.

\todo{write and explain the ELBO/learning objective for the Bayes $\ve{\theta}$ case}